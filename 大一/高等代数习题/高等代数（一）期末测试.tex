\documentclass[12pt, a4paper, oneside,UTF8]{ctexart}
\usepackage{amsmath, amsthm, amssymb, bm, color, framed, graphicx, hyperref, mathrsfs,amsfonts}
\usepackage[left=1.0cm,right=1.0cm,top=1.5cm,bottom=1.5cm]{geometry}
%\linespread{1.5}
\definecolor{shadecolor}{RGB}{255, 255, 255}
\newcounter{problemname}
\newenvironment{problem}{\begin{shaded}\stepcounter{problemname}\par\noindent\textbf{题目\arabic{problemname}. }}{\end{shaded}\par}
\newenvironment{solution}{\par\noindent\textbf{解答. }}{\par}
\newenvironment{note}{\par\noindent\textbf{题目\arabic{problemname}的注记. }}{\par}
\ctexset{ section = { format={\Large \bfseries } } }

\title{高等代数(一)期末考试题}
\author{李冠霖}
\date{\today}
\begin{document}
\maketitle
\begin{problem}
已知$f(x) | f(x^{n})$,求证,$f(x)$的根只能是0或者单位根
\end{problem}
\begin{problem}
计算行列式
\begin{equation*}
    \left|\begin{matrix}
        a_{1}  & b_{1}  & 0      & 0      & \cdots & 0      \\
        0      & a_{2}  & b_{2}  & 0      & \cdots & 0      \\
        0      & 0      & a_{3}  & b_{3}  & \cdots & 0      \\
        \vdots & \vdots & \vdots & \vdots & \ddots & \vdots \\
        b_{n}  & 0      & 0      & 0      & \cdots & a_{n}
    \end{matrix}\right|
\end{equation*}
\begin{note}
    其实原题还有一个转置的这个行列式,不过本人比较懒,不想打了,各位看官自己将就将就QAQ
\end{note}
\end{problem}
\begin{problem}
求证向量组$\alpha_{1},\alpha_{2},\alpha_{3},\ldots,\alpha_{s}$线性无关等价于$\beta - \alpha_{1},\beta - \alpha{2}, \ldots , \beta - \alpha_{s} $ 线性无关,其中$\beta = \sum_{i = 1}^{n} \limits \alpha_{i}$
\end{problem}
\begin{problem}
二次曲线过点$(0,0)(1,1)(2,1)(1,4)(1,0) $,求二次曲线的方程
\end{problem}
\begin{problem}
$A^{*}$是矩阵$
    \left[\begin{matrix}
            a_{11} & a_{12} & a_{13} \\
            a_{21} & a_{22} & a_{23} \\
            a_{31} & a_{32} & a_{33}
        \end{matrix}\right]
$的伴随矩阵,$|A| = 2$,矩阵$B =
    \left[\begin{matrix}
            -a_{11} & a_{12} & a_{13} \\
            -a_{21} & a_{22} & a_{23} \\
            -a_{31} & a_{32} & a_{33}
        \end{matrix}\right]$,求$A^{*}B$
\end{problem}
\begin{problem}
矩阵$A =
    \left[\begin{matrix}
            1 & -2 & 0 \\
            1 & 2  & 0 \\
            0 & 0  & 2
        \end{matrix}\right]$,矩阵$B$为三阶方阵,且$2B^{-1}A + 4E = A $证明$B - 2E $可逆,并求出${(B - 2E)}^{-1} $
\end{problem}
\end{document}

